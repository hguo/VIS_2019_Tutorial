%\documentclass[journal]{vgtc}                % final (journal style)
%\documentclass[review,journal]{vgtc}         % review (journal style)
%\documentclass[widereview]{vgtc}             % wide-spaced review
\documentclass[preprint,journal]{vgtc}       % preprint (journal style)
%\documentclass[electronic,journal]{vgtc}     % electronic version, journal

%% Uncomment one of the lines above depending on where your paper is
%% in the conference process. ``review'' and ``widereview'' are for review
%% submission, ``preprint'' is for pre-publication, and the final version
%% doesn't use a specific qualifier. Further, ``electronic'' includes
%% hyperreferences for more convenient online viewing.

%% Please use one of the ``review'' options in combination with the
%% assigned online id (see below) ONLY if your paper uses a double blind
%% review process. Some conferences, like IEEE Vis and InfoVis, have NOT
%% in the past.

%% Please note that the use of figures other than the optional teaser is not permitted on the first page
%% of the journal version.  Figures should begin on the second page and be
%% in CMYK or Grey scale format, otherwise, colour shifting may occur
%% during the printing process.  Papers submitted with figures other than the optional teaser on the
%% first page will be refused.

%% These three lines bring in essential packages: ``mathptmx'' for Type 1
%% typefaces, ``graphicx'' for inclusion of EPS figures. and ``times''
%% for proper handling of the times font family.

\usepackage{mathptmx}
\usepackage{graphicx}
\usepackage{times}
\usepackage{float}
\usepackage[style=ieee,backend=biber,defernumbers=true]{biblatex}

%% We encourage the use of mathptmx for consistent usage of times font
%% throughout the proceedings. However, if you encounter conflicts
%% with other math-related packages, you may want to disable it.

%% This turns references into clickable hyperlinks.
\usepackage[bookmarks,backref=true,linkcolor=black]{hyperref} %,colorlinks
\hypersetup{
  pdfauthor = {},
  pdftitle = {},
  pdfsubject = {},
  pdfkeywords = {},
  colorlinks=true,
  linkcolor= black,
  citecolor= black,
  pageanchor=true,
  urlcolor = black,
  plainpages = false,
  linktocpage
}

%% If you are submitting a paper to a conference for review with a double
%% blind reviewing process, please replace the value ``0'' below with your
%% OnlineID. Otherwise, you may safely leave it at ``0''.
\onlineid{0}

%% declare the category of your paper, only shown in review mode
\vgtccategory{Research}

%% allow for this line if you want the electronic option to work properly
\vgtcinsertpkg

%% In preprint mode you may define your own headline.
\preprinttext{}%To appear in an IEEE VGTC sponsored conference.}

%% Paper title.
\title{Distribution-driven Data Representation, Visualization, and \\ Uncertainty Analysis}

%% This is how authors are specified in the journal style

%% indicate IEEE Member or Student Member in form indicated below
\author{Soumya Dutta, Hanqi Guo, Hans-Christian Hege, and Han-Wei Shen}
\authorfooter{
%% insert punctuation at end of each item
\item
 Soumya Dutta is with Los Alamos National Laboratory. \\E-mail: sdutta@lanl.gov.
\item
 Hanqi Guo is with Argonne National Laboratory. Email: hguo@anl.gov.
\item
 Hans-Christian Hege is with Zuse Institute Berlin. Email: hege@zib.de.
\item
 Han-Wei Shen is with The Ohio State University. \\Email: shen.94@osu.edu.
}

%other entries to be set up for journal
% \shortauthortitle{Biv \MakeLowercase{\textit{et al.}}: Global Illumination for Fun and Profit}
%\shortauthortitle{Firstauthor \MakeLowercase{\textit{et al.}}: Paper Title}

%% Uncomment below to disable the manuscript note
%\renewcommand{\manuscriptnotetxt}{}

%% Copyright space is enabled by default as required by guidelines.
%% It is disabled by the 'review' option or via the following command:
% \nocopyrightspace

%%%%%%%%%%%%%%%%%%%%%%%%%%%%%%%%%%%%%%%%%%%%%%%%%%%%%%%%%%%%%%%%
%%%%%%%%%%%%%%%%%%%%%% START OF THE PAPER %%%%%%%%%%%%%%%%%%%%%%
%%%%%%%%%%%%%%%%%%%%%%%%%%%%%%%%%%%%%%%%%%%%%%%%%%%%%%%%%%%%%%%%%

\newcommand{\addverticalspace}{\vspace{3mm}}

\bibliography{template}
\nocite{*}

\DeclareBibliographyCategory{Dutta}
\addtocategory{Dutta}{}

\DeclareBibliographyCategory{Guo}
\addtocategory{Guo}{}

\DeclareBibliographyCategory{Hege}
\addtocategory{Hege}{}

\DeclareBibliographyCategory{Shen}
\addtocategory{Shen}{}

% \DeclareFieldFormat{labelnumberwidth}{}
% \setlength{\biblabelsep}{0pt}


\begin{document}

%% The ``\maketitle'' command must be the first command after the
%% ``\begin{document}'' command. It prepares and prints the title block.

%% the only exception to this rule is the \firstsection command

\maketitle

\section*{TITLE}
Distribution-driven Data Representation, Visualization, and \\ Uncertainty Analysis.

\section*{DURATION}
The tutorial is a proposal of a half-day tutorial, including 180 minutes presentation, 20 minutes discussion, and 20 minutes coffee break.

\section*{SCHEDULE}

\vspace{-0.1in}
\begin{table}[H]
\begin{tabular}{lll}
Introduction & Soumya Dutta & 10 minutes\\
Talk 1 & Han-Wei Shen & 45 minutes\\
Talk 2 & Soumya Dutta & 45 minutes\\
Coffee Break & -------- & 20 minutes\\
Talk 3 & Hans-Christian Hege & 45 minutes\\
Talk 4 & Hanqi Guo & 45 minutes\\
Conclusion & Hans-Christian Hege & 10 minutes
\end{tabular}
\end{table}

\section*{ORGANIZERS}

\vspace{-0.1in}
\begin{table}[H]
\begin{tabular}{ll}
Soumya Dutta & Los Alamos National Laboratory\\
Hanqi Guo & Argonne National Laboratory\\
Hans-Christian Hege & Zuse Institute Berlin\\
Han-Wei Shen & The Ohio State University\\
\end{tabular}
\end{table}

\section*{ABSTRACT}
Efficient analysis and visualization of data using statistical methods have been a popular choice in the visualization community for many years. As the size of data has grown rapidly, researchers have adopted for techniques, which are primarily driven by efficient identification and analysis of important regions (generally termed as features) in the data, instead of looking at the data at its entirety. Distributions estimated from data samples are capable of compactly representing the statistical characteristics of features, which can be efficiently analyzed and visualized. Recent developments in the visualization domain have demonstrated the wide  applicability of distribution-based data visualization methods by introducing novel stochastic algorithms and addressing problems such as: feature identification, extraction, and tracking; multi-variable relationship exploration; query-driven visualization; in situ data reduction; integral curve-based flow feature analysis; transfer function design and and many more.

Besides being able to compactly represent statistical data properties, a fundamental advantage of distribution-based data analysis techniques is the capability of uncertainty quantification during visualization. Uncertainty-aware visualization algorithms developed from distribution-based data representations can successfully convey the trustworthiness of the visual representation to the application scientists so that they can draw meaningful conclusions from the results about their data. Furthermore, as we step into the era of big data, the relevance of distribution-based methods has become even more prominent since distribution-based data summaries have been shown to be significantly smaller than the full-resolution raw data and such data reduction can be performed in the in situ environment. As a result, a wide range of visualization applications using distribution-based data has been developed already which evidently indicate that distribution-based methods will provide a promising path forward for the future data visualization researches.

With the aforementioned promising aspects of distribution-based methods in visual analysis and uncertainty quantification, we propose to organize a half-day tutorial on the topic of distribution-driven data representation, visualization, and uncertainty analysis. The tutorial will highlight concepts related to distribution-based data modeling techniques appropriate for a variety of applications. A comprehensive discussion on many state-of-the-art visualization methods that uses distribution data will be carried out which will serve as a groundwork for the audience who are interested in conducting research in this avenue. We will systematically introduce different categories of visual-analytics tasks that have utilized and benefited from distributions. Next, concepts and applications of uncertainty-aware visualization techniques will be presented which are built on top of distribution-based data. Finally, the latest research trends and applications adopting distribution-guided data analysis for both in situ and post-hoc data exploration will be discussed. We will conclude our session by highlighting several future scopes and open problems in this area needed to be solved for the further advancements of distribution-based methods in visualization.

\section*{LEVEL}
Intermediate/Advanced.

\section*{PREREQUISITE}
A general understanding of probability, statistics, and statistical distributions, including basic knowledge of uncertainty quantification in visualization.

% \section*{AUDIENCE}
% The intended audience includes students, researchers and practitioners who are interested in the recent advances in flow visualization. More details to be added.

\section*{DESCRIPTION}
The brief outline of each section is presented below:

\addverticalspace

\noindent\textbf{\textit{Distribution-based Data Representation and Modeling}}\\
\textbf{Han-Wei Shen}
\paragraph{Abstract}
Please provide a small abstract for this section.

\addverticalspace

\noindent\textbf{\textit{Stochastic Algorithms for Visual Analysis}}\\
\textbf{Soumya Dutta}
\paragraph{Abstract}
Please provide a small abstract for this section.

\addverticalspace

\noindent\textbf{\textit{Uncertainty Quantification and Visualization}}\\
\textbf{Hans-Christian Hege}
\paragraph{Abstract}
Please provide a small abstract for this section.

\addverticalspace

\noindent\textbf{\textit{Application of Distributions in Visualization}}\\
\textbf{Hanqi Guo}
\paragraph{Abstract}

This talk presents applications of distribution-driven visualizations in understanding uncertain unsteady flows.  The goal of this study is to understand transport behavior in uncertain time-varying flow fields by redefining the
finite-time Lyapunov exponent (FTLE) and Lagrangian coherent structure (LCS) as stochastic counterparts of their traditional deterministic definitions~\cite{Guo16}. Three new concepts are introduced: the distribution of the FTLE (D-FTLE), the FTLE of distributions (FTLE-D), and uncertain LCS (U-LCS). The D-FTLE is the probability density function of FTLE values for every spatiotemporal location, which can be visualized with different statistical measurements. The FTLE-D extends the deterministic FTLE by measuring the divergence of particle distributions. It gives a statistical overview of how transport behaviors vary in neighborhood locations. The U-LCS, the probabilities of finding LCSs over the domain, can be extracted with stochastic ridge finding and density estimation
algorithms.  

In addition, this talk will also cover scalability issues of estimating uncertain transport behaviors---stochastic flow maps (SFMs)---for visualizing and analyzing uncertain unsteady flows~\cite{Guo19}. Computing flow maps from uncertain flow fields is extremely expensive because it requires many Monte Carlo runs to trace densely seeded particles in the flow. We reduce the computational cost by decoupling the time dependencies in SFMs so that we can process shorter sub time intervals independently and then compose them together for longer time periods. Adaptive refinement is also used to reduce the number of runs for each location.  We parallelize over tasks---packets of particles in our design---to achieve high efficiency in MPI/thread hybrid programming. Such a task model also enables CPU/GPU coprocessing. We show the scalability on two supercomputers, Mira (up to 256K Blue Gene/Q cores) and Titan (up to 128K Opteron cores and 8K GPUs), that can trace billions of particles in seconds.


\addverticalspace

\section*{TUTORIAL NOTES}

The tutorial notes will contain copies of the presented slides for all the talks, description of the tutorial,  bibliographies used during the tutorial as well as a general set of references related to the tutorial.

\section*{SPEAKERS}
The background of each speaker is provided in alphabetical order.

\addverticalspace

\noindent \textbf{1. Soumya Dutta}\\
\emph{Los Alamos National Laboratory}\\
\href{mailto:sdutta@lanl.gov}{sdutta@lanl.gov}\\
\url{https://sites.google.com/view/soumyadutta/}

\addverticalspace

Soumya Dutta is currently a postdoctoral research associate in Data Science at Scale team at Los Alamos National Laboratory. He received his MS and Ph.D. degree in Computer Science and Engineering from the Ohio State University in May 2017 and May 2018 respectively. His major research interests are statistical data summarization and analysis; in situ data analysis, reduction, and feature exploration; uncertainty analysis; and time-varying, multivariate data exploration. He received an honorable mention award at IEEE Visualization (SciVis track) 2016 and was nominated for the Presidential Fellowship from the Department of Computer Science and Engineering at the Ohio State University in 2017. He won best poster award in the Annual CSE Student Research Poster Exhibition at the Ohio State University in 2017 and 2018.

\printbibliography[title={Relevant Publications},category=Dutta]

\addverticalspace

\noindent \textbf{2. Hanqi Guo}\\
\emph{Argonne National Laboratory}\\
\href{mailto:hguo@anl.gov}{hguo@anl.gov}\\
\url{http://www.mcs.anl.gov/~hguo/}

\addverticalspace

Hanqi Guo is an assistant computer scientist in the Mathematics and Computer Science Division at Argonne National Laboratory.  He received his PhD degree in computer science from Peking University in 2014, and the BS degree in mathematics and applied mathematics from Beijing University of Posts and Telecommunications in 2009. His research interests are mainly on uncertainty visualization, flow visualization, and large-scale scientific data visualization.

\printbibliography[title={Relevant Publications},category=Guo]

\addverticalspace

\noindent \textbf{3. Hans-Christian Hege}\\
\emph{Zuse Institute Berlin}\\
\href{mailto:hege@zib.de}{hege@zib.de}\\
\url{https://www.zib.de/hege/}

\addverticalspace

Hans-Christian Hege is a ...

\printbibliography[title={Relevant Publications},category=Hege]


\noindent \textbf{4. Han-Wei Shen}\\
\emph{The Ohio State University}\\
\href{mailto:shen.94@osu.edu}{shen.94@osu.edu}\\
\url{http://web.cse.ohio-state.edu/~shen.94/shen.94/Welcome.html}

\addverticalspace
Han-Wei Shen is a full professor at The Ohio State University. He received his BS degree from Department of Computer Science and Information Engineering at National Taiwan University in 1988, the MS degree in computer science from the State University of New York at Stony Brook in 1992, and the PhD degree in computer science from the University of Utah in 1998. From 1996 to 1999, he was a research scientist at NASA Ames Research Center in Mountain View California. His primary research interests are scientific visualization and computer graphics. Professor Shen is a winner of National Science Foundation's CAREER award and US Department of Energy's Early Career Principal Investigator Award. He also won the Outstanding Teaching award twice in the Department of Computer Science and Engineering at the Ohio State University.

\printbibliography[title={Relevant Publications},category=Shen]

\addverticalspace



% \bibliographystyle{abbrv}
% %%use following if all content of bibtex file should be shown
% %\nocite{*}
% \bibliography{template}
\end{document}